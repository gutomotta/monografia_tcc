\chapter{Introdução}
\label{cap:introducao}

\section{Motivação e objetivos}
    Dos diversos elementos que compõem a música tonal, a harmonia - cujo componente elementar, ao menos dentro da tradição da música ocidental, é o acorde - é um de suma importância. O reconhecimento de acordes é um problema bastante estudado dentro da área de Recuperação da Informação Musical (MIR) e diversos trabalhos já foram apresentados na academia~\cite{harte,muller}.
    
    Algoritmos de reconhecimento de acordes num sinal de áudio podem ser úteis em diversos problemas, como em classificação de gêneros, por exemplo, ou, de forma particularmente relevante para a escolha do tema deste trabalho, na produção automática de cifras.
    
    Atualmente já existem aplicações capazes de fazer o reconhecimento de acordes de forma eficiente, como o \href{http://clam-project.org/wiki/Chordata_tutorial}{Chordata}\footnote{\url{http://clam-project.org/wiki/Chordata_tutorial}}, do \href{http://clam-project.org/index.html}{projeto CLAM}\footnote{\url{http://clam-project.org/index.html}}, ou o \href{https://chordify.net/}{Chordify}\footnote{\url{https://chordify.net/}}.
    
    %%Obs: incluí os footnotes pois quem estiver lendo o texto impresso não saberia que existem links escondidos aqui.
    
    Neste cenário, mostrou-se desejável familiarizar-se com o problema e conhecer suas possíveis soluções. Como referência para o estudo, utilizou-se o trabalho introdutório de \cite{muller}, que apresenta duas soluções diferentes. Restringiu-se, no contexto deste trabalho, a uma das soluções: o algoritmo de classificação de acordes baseado em templates, acompanhado de técnicas para aperfeiçoamento do mesmo.

\section{Estrutura da monografia}
    Além dos capítulos introdutório e conclusivo, o conteúdo desta monografia está dividido em dois capítulos: Fundamentação Teórica e Desenvolvimento.
    
    No capítulo de Fundamentação Teórica se discorrerá brevemente sobre os conceitos necessários para o entendimento do problema e da solução apresentada neste trabalho. Esses conceitos vêm tanto da teoria musical quanto da ciência da computação. Em especial, se definirá o conceito de acorde, que é fundamental neste contexto, e diversos conceitos secundários necessários para sua definição. Se apresentarão também conceitos da área de Processamento de Sinais Digitais como sinal de áudio, Transformada de Fourier, espectrograma e cromagrama.
    
    No capítulo de Desenvolvimento, será proposto um algoritmo para a resolução do problema de reconhecimento de acordes, e se discorrerá sobre suas limitações e técnicas para melhorar sua eficácia. Também será discutida uma metodologia de avaliação do algoritmo e apresentada uma base de anotações de referência. Por último, serão relatados os experimentos feitos durante este trabalho e os resultados e conclusões decorrentes deles.
    