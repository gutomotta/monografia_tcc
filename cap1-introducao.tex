\chapter{Introdução}
\label{cap:introducao}
    Esta é uma monografia parcial. Em algumas partes deste texto, estão descritos trechos que serão redigidos na monografia final no sentido de torná-lo mais completo e autocontido.
    
    Além das mudanças já planejadas, pretende-se aperfeiçoar o texto em geral, acrescentando ou removendo detalhes conforme for desejável em cada trecho.
    
    A seção de experimentos desta versão do texto, em particular, apenas descreve a metodologia usada nos experimentos deste trabalho, e apresenta os dados dos resultados de forma meramente expositiva. Tais resultados serão discutidos de forma explicativa na versão final da monografia, onde também se disponibilizarão os dados agregados e com melhores visualizações.
    
    Enfim, diversos itens serão revisados até a monografia final - espera-se, porém, que esta versão passe uma visão geral do trabalho e dos planos para a versão final, que conterá um relato completo.

\section{Motivação e objetivos}
    Dos diversos elementos que compõem a música tonal, a harmonia - cujo componente elementar, ao menos dentro da tradição da música ocidental, é o acorde - é um de suma importância. O reconhecimento de acordes é um problema bastante estudado dentro da área de Recuperação da Informação Musical (MIR) e diversos trabalhos já foram apresentados na academia.
    
    Algoritmos de reconhecimento de acordes num sinal de áudio podem ser úteis em diversos problemas, como em classificação de gêneros, por exemplo, ou, de forma particularmente relevante para a escolha do tema deste trabalho, na produção automática de cifras.
    
    Hoje em dia já existem aplicativos em produção capazes de fazer o reconhecimento de acordes de forma eficiente, como o \href{http://clam-project.org/wiki/Chordata_tutorial}{Chordata}, do \href{http://clam-project.org/index.html}{projeto CLAM}, ou o \href{https://chordify.net/}{Chordify}.
    
    Neste cenário, mostrou-se desejável familiarizar-se com o problema, e conhecer possibilidades de solução do mesmo. Como referência para o estudo, utilizou-se o trabalho introdutório de \cite{muller}, que apresenta duas soluções diferentes. Restringiu-se, no contexto deste trabalho, a uma das soluções: o algoritmo de classificação de acordes baseado em templates, acompanhado de técnicas para aperfeiçoamento do mesmo.

%\section{Trabalhos relacionados}
    % Comentário sobre o parágrafo

\section{Estrutura da monografia}
    % Comentário sobre o parágrafo
    Além do capítulo introdutório e conclusivo, o conteúdo deste monografia está dividido em dois capítulos: Fundamentação Teórica e Desenvolvimento.
    
    No capítulo de Fundamentação Teórica se discorrerá brevemente sobre os conceitos necessários para o entendimento do problema e da solução apresentada neste trabalho. Esses conceitos vêm tanto da teoria musical quanto da ciência da computação. Em especial, se definirá o conceito de acorde, que é fundamental neste contexto, e diversos conceitos secundários necessários para a definição do mesmo. Se apresentarão também conceitos da área de Processamento de Sinais Digitais como sinal de áudio, Transformada de Fourier, espectrograma e cromagrama.
    
    No capítulo de Desenvolvimento, será proposto um algoritmo para a resolução do problema de reconhecimento de acordes, e se discorrerá sobre limitações do mesmo e técnicas para melhorar sua eficácia. Também será discutida uma metodologia de avaliação do algoritmo e apresentada uma base de anotações de referência. Por último, serão relatados os experimentos feitos durante este trabalho e os resultados e conclusões decorrentes deles.
    