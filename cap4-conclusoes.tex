\chapter{Conclusões}
\label{cap:conclusoes}

    Neste trabalho, se estudou o problema de reconhecimento de acordes e uma solução particular para ele: a classificação baseada em templates. Se discutiram pontos do algoritmo em que se poderia experimentar técnicas para aperfeiçoá-lo. Se descreveu um método de avaliação do algoritmo que foi usado experimentalmente para decidir se as técnicas de aperfeiçoamento discutidas apresentariam ou não melhorias na prática.
    
    Dadas as simplificações adotadas, consideraram-se satisfatórios os resultados obtidos, que alcançaram uma precisão média de ${52}\%$ numa base de 180 fonogramas. Para algumas das possíveis técnicas de aperfeiçoamento descritas, não se pode observar experimentalmente uma melhoria nos resultados, como no caso do uso de templates aprendidos a partir de anotações de referência; para outras, pode-se observar um ganho significativo: como no caso da suavização temporal de cromas e no uso da transformada Q-constante em vez da transformada de Fourier, técnicas que se conclui serem essenciais para construir um algoritmo classificador de acordes baseado em templates.
    
    Muitos caminhos ainda podem ser explorados no estudo de reconhecimento de acordes. O algoritmo apresentado não classifica intervalos do sinal sem acordes, e usa um conjunto pequeno de acordes passíveis de classificação - diversos acordes poderiam ser incluídos. Além disso, se poderia estudar algoritmos que fazem uso de técnicas de aprendizado de máquina para aperfeiçoar a classificação.
